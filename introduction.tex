% -----------------------------------------------------------------------
% -----------------------------------------------------------------------
% -----------------------------------------------------------------------
% Einleitung
% -----------------------------------------------------------------------
% -----------------------------------------------------------------------
% -----------------------------------------------------------------------
\chapter{Introduction}

Embarking on the literary journey of a captivating book, one often finds themselves entangled in a web of intricate characters, each contributing to the rich tapestry of the narrative. Yet, the overwhelming abundance of details can pose a challenge, making it difficult for readers to retain a comprehensive understanding of each character's essence. To address this challenge, this thesis aims to employ Retrieval-Augmented Generation (RAG) in order to create improved, meticulous and detailed personal descriptions for each character.\\

An important aspect of this study involves compiling existing literature and human-written characterizations of its characters, allowing for a comparison between these established descriptions and the results generated by large language models (LLMs) when given various prompts created using different techniques that extract information from the according literatur. The main focus will be on the methods for extracting and modifying information from literatur in order to enhance the query results, along with their evaluation and analysis.\\

The findings of this thesis can be significantly applied to similar Natural Language Processing (NLP) tasks, especially those requiring the extraction and summarization of information from running literature into different text formats. It is particularly interesting to observe how LLMs, such as LLaMA, respond to slight variations in query formulations. 
Ultimately it can also be used to improve the readers comprehension and as an educational tool that can be used to specifically extract information about a certain entitiy from a text.\\


I intend to structure the thesis around the experiments and their results. I believe that segregating the experiments with formal sections would disrupt the central theme, making it harder for both me and the reader to follow. Therefore, I will provide a broad overview of the methods I might consider using in the methodology section and then each experiment will be discussed and evaluated individually.
