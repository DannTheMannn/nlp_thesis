\chapter{Experiments}

All my experiments have been conducted partially on my own computer but also over ssh on a remote server from the LT group at the University of Hamburg. This was mainly due to accessing a better GPU like the NVIDEA RTX A6000 for doing more computational intensive work such as prompting large language models (LLMs) and generating embeddings.


\section{Base Experiment}

For my first experiment, I formulated four prompts with slightly different wordings to observe how varying prompts affect the outcomes of the LLM. For each prompt, I tested two versions: one with additional passages from the literature providing information about the character, and one without such information, requiring the model to rely solely on its training data. All the eight raw prompts \ref*{fig:prompt2} contain tags. These tags will be interpreted as following. ``[INST]'' and ``[\textbackslash INST]'' mark the start and end of each query instruction. ``\{character\}'' and ``\{book\}'' will be replaced with the real character name and book title. ``\{passages\}'' marks the spot where a collection of retrieved passages from the book for the given character that should help the Llama model with its characterizations will be passed into the prompt.\\


\begin{figure}
    \begin{center}
        \begin{tabular}{|c|m{10cm}|}
            \hline
            Prompt      & Instruction                                                                                                                                                 \\ [0.5ex]
            \hline\hline
            $P_{1}^{z}$ & "[INST]Write a summary about the character \{character\} in the book \{book\}.[/INST]"                                                                      \\
            \hline
            $P_{1}^{r}$ & "[INST]Write a summary about the character \{character\} in the given text passages: \textbackslash n \{passages\}[/INST]"                                  \\
            \hline\hline
            $P_{2}^{z}$ & "[INST]Write a summary in the style of a fandom article about the character \{character\} in the book \{book\}.[/INST]"                                     \\
            \hline
            $P_{2}^{r}$ & "[INST]Write a summary in the style of a fandom article about the character \{character\} in the given text passages: \textbackslash n \{passages\}[/INST]" \\
            \hline\hline
            $P_{3}^{z}$ & "[INST]Provide a concise overview of the character \{character\} from the book \{book\}.[/INST]"                                                            \\
            \hline
            $P_{3}^{r}$ & "[INST]Provide a concise overview of the character \{character\} based on the following excerpts: \textbackslash n \{passages\}[/INST]"                     \\
            \hline\hline
            $P_{4}^{z}$ & "[INST]rite sumary bout thee cara cter \{character\} of th book \{book\}.[/INST]"                                                                           \\
            \hline
            $P_{4}^{r}$ & "[INST]rite sumary bout thee cara cter \{character\} bsed th fllowing excerpts: \textbackslash n \{passages\}[/INST]"                                       \\ [1ex]
            \hline
        \end{tabular}
    \end{center}
    \label{fig:prompts2}
\end{figure}





As you can see, $P_{2}$ is more specific, requesting the style of a fandom article, whereas $P_{3}$ is less precise, asking only for an overview without specifying a particular format. The last prompt $P_{4}$ is similar to $P_{1}$ but is intentionally faulty by missing characters.These different prompts are used to determine the overall effects of various prompt wordings and faulty instructions on the language model.



In this first experiment, I selected additional information from the book by filtering for every sentence in which the character's name occurred at least once. Since the number of tokens might exceed the maximum input size of the LLaMA model, I removed every $n$-th sentence, where $n$ is calculated in such a way that the query size fits perfectly.
Additionally, because characters are more likely to be introduced in the first sentences where they appear in the book, I added an additional cutoff $\alpha$. This cutoff represents the percentage of relevant sentences (with character name occurrences) to which every sentence with name occurence will be taken, so the rule of taking every $n$-th sentence only affects sentences after the cutoff. Overall the passage retrieval for this experiment $R_{base}$ works as follows. Let $S = \{s_i \mid 1 \leq i \leq k \}$ be the set of size $k$ which contains all relevant sentences containig the character and $l$ be the maximum inputsize of the Llama query. We first definde a function $S_{t}(a, b) = \{ s_{ti} \mid a \cdot k \leq ti \leq b \cdot k \}$, that enables a range selection of sentences with a lower and upper limit and a parameter $t$ for the stepsize. If we now choose our $n$ the right way
\[n = \begin{cases}
        \left\lfloor \frac{k - \alpha k}{l}\right\rfloor & \text{if } k - \alpha k > l \\
        1                                                & \text{otherwise}
    \end{cases} \] we can write $R_{base}$ as \[R_{base} = S_{1}(0, \alpha) \cup S_{n} ( \alpha, 1) \].
I utilized the Mixtral 7B model with quantized weights to process the prompts. Quantization is a technique designed to reduce the computational and memory demands of running inference by using low-precision data types, such as 8-bit integers (int8), instead of the standard 32-bit floating-point (float32). This approach decreases memory storage requirements, theoretically lowers energy consumption, and accelerates operations like matrix multiplication through integer arithmetic. Moreover, it enables models to operate on embedded devices that may only support integer data types. I started with the smallest quantization level, $Q2_K$ weights, to achieve the quickest responses. For evaluation, I used BLEUScore and BERTScore to compare the generated results against manually written articles from fandom.com.\\

For the analysis of the results, I decided to use boxplots, t-tests, and Spearman correlation. To quickly summarize, a paired t-test compares the means of two related groups to determine if there is a statistically significant difference between these means. I used them to have quantitative proof that the results improved after passage retrieval. A boxplot segregates the data into four parts by determining the three quartiles. Additionally, with the added histograms, it allows a better overview of the distribution of the data than just looking at the mean or median. The Spearman correlation evaluates monotonic relationships between two ranked variables, which helps to identify the general tendency of the passage retrieval.

\subsection{Results}
\begin{figure}[H]
    \centering
    \makebox[\textwidth][c]{
        \begin{minipage}{1.1\textwidth}
            \centering
            \begin{subfigure}[b]{0.45\textwidth}
                \centering
                \begin{tikzpicture}
                    % Main axis (scatter plot)
                    \begin{axis}[
                            enlargelimits=false,
                            title={Prompt 1},
                            title style={at={(0.2,1.2)}, anchor=north},
                            xlabel={$BERT(P_{1}^{z})$},
                            ylabel={$BERT(P^{r})$},
                            xmin=0.35, xmax=0.65,
                            ymin=0.35, ymax=0.65,
                            xtick distance=0.1, ytick distance=0.1,
                            name=main axis,
                            width=\textwidth,
                            height=\textwidth
                        ]
                        \addplot[color=black, thick, dotted] {x};
                        \addplot+[
                            color=blue,
                            only marks,
                            mark=o,
                            mark size=1.0pt
                        ] table [col sep=comma, x=F1, y=wF1] {ressources/data/results.csv};
                    \end{axis}

                    % Histogram for the x axis
                    \begin{axis}[
                            anchor=south west,
                            at={(main axis.north west)},
                            height=3cm,
                            width=\textwidth,
                            xtick=\empty,
                            ytick=\empty,
                            axis x line*=bottom,
                            axis y line*=left,
                            axis line style={draw=none},
                            xmin=0.35, xmax=0.65
                        ]
                        \addplot [
                            hist={data=x, bins=30}, % Increase number of bins for better resolution % Use ybar interval
                            fill=gray!50
                        ] table [col sep=comma, x=F1, y=wF1] {ressources/data/results.csv};
                    \end{axis}

                    % Histogram for the y axis
                    \begin{axis}[
                            anchor=north west,
                            at={(main axis.north east)},
                            width=3cm,
                            height=\textwidth,
                            xtick=\empty,
                            ytick=\empty,
                            axis x line*=top,
                            axis y line*=right,
                            axis line style={draw=none},
                            ymin=0.35, ymax=0.65
                        ]
                        \addplot [
                            hist={data min=0.35, data max=0.65, handler/.style={xbar interval}, bins=30}, % Use ybar interval
                            fill=gray!50,
                            x filter/.code=\pgfmathparse{rawy}, % Interpret the x values of the histogram as y values 
                            y filter/.code=\pgfmathparse{rawx} % And vice versa
                        ] table [col sep=comma, x=F1, y=wF1] {ressources/data/results.csv};
                    \end{axis}
                \end{tikzpicture}
                \label{fig:prompt1}
            \end{subfigure}
            \hfill
            \begin{subfigure}[b]{0.45\textwidth}
                \centering
                \begin{tikzpicture}
                    % Main axis (scatter plot)
                    \begin{axis}[
                            enlargelimits=false,
                            title={Prompt 2},
                            title style={at={(0.2,1.2)}, anchor=north},
                            xlabel={$BERT(P_{2}^{z})$},
                            % ylabel={$BLEU(P')$},
                            xmin=0.35, xmax=0.65,
                            ymin=0.35, ymax=0.65,
                            xtick distance=0.1, ytick distance=0.1,
                            name=main axis,
                            width=\textwidth,
                            height=\textwidth
                        ]
                        \addplot[color=black, thick, dotted] {x};
                        \addplot+[
                            color=blue,
                            only marks,
                            mark=o,
                            mark size=1.0pt
                        ] table [col sep=comma, x=F1, y=wF1] {ressources/data/results2.csv};
                    \end{axis}

                    % Histogram for the x axis
                    \begin{axis}[
                            anchor=south west,
                            at={(main axis.north west)},
                            height=3cm,
                            width=\textwidth,
                            xtick=\empty,
                            ytick=\empty,
                            axis x line*=bottom,
                            axis y line*=left,
                            axis line style={draw=none},
                            xmin=0.35, xmax=0.65
                        ]
                        \addplot [
                            hist={data=x, bins=30}, % Increase number of bins for better resolution, % Use ybar interval
                            fill=gray!50
                        ] table [col sep=comma, x=F1, y=wF1] {ressources/data/results2.csv};
                    \end{axis}

                    % Histogram for the y axis
                    \begin{axis}[
                            anchor=north west,
                            at={(main axis.north east)},
                            width=3cm,
                            height=\textwidth,
                            xtick=\empty,
                            ytick=\empty,
                            axis x line*=top,
                            axis y line*=right,
                            axis line style={draw=none},
                            ymin=0.35, ymax=0.65
                        ]
                        \addplot [
                            hist={data min=0.35, data max=0.65, handler/.style={xbar interval}, bins=30}, % Use ybar interval
                            fill=gray!50,
                            x filter/.code=\pgfmathparse{rawy}, % Interpret the x values of the histogram as y values 
                            y filter/.code=\pgfmathparse{rawx} % And vice versa
                        ] table [col sep=comma, x=F1, y=wF1] {ressources/data/results2.csv};
                    \end{axis}
                \end{tikzpicture}
                \label{fig:prompt2}
            \end{subfigure}
            \vspace{0.5cm}
            \begin{subfigure}[b]{0.45\textwidth}
                \centering
                \begin{tikzpicture}
                    % Main axis (scatter plot)
                    \begin{axis}[
                            enlargelimits=false,
                            title={Prompt 3},
                            title style={at={(0.2,1.2)}, anchor=north},
                            xlabel={$BERT(P_{3}^{z})$},
                            ylabel={$BERT(P^{r})$},
                            xmin=0.35, xmax=0.65,
                            ymin=0.35, ymax=0.65,
                            xtick distance=0.1, ytick distance=0.1,
                            name=main axis,
                            width=\textwidth,
                            height=\textwidth
                        ]
                        \addplot[color=black, thick, dotted] {x};
                        \addplot+[
                            color=blue,
                            only marks,
                            mark=o,
                            mark size=1.0pt
                        ] table [col sep=comma, x=F1, y=wF1] {ressources/data/results3.csv};
                    \end{axis}

                    % Histogram for the x axis
                    \begin{axis}[
                            anchor=south west,
                            at={(main axis.north west)},
                            height=3cm,
                            width=\textwidth,
                            xtick=\empty,
                            ytick=\empty,
                            axis x line*=bottom,
                            axis y line*=left,
                            axis line style={draw=none},
                            xmin=0.35, xmax=0.65
                        ]
                        \addplot [
                            hist={data=x, bins=30}, % Increase number of bins for better resolution, % Use ybar interval
                            fill=gray!50
                        ] table [col sep=comma, x=F1, y=wF1] {ressources/data/results3.csv};
                    \end{axis}

                    % Histogram for the y axis
                    \begin{axis}[
                            anchor=north west,
                            at={(main axis.north east)},
                            width=3cm,
                            height=\textwidth,
                            xtick=\empty,
                            ytick=\empty,
                            axis x line*=top,
                            axis y line*=right,
                            axis line style={draw=none},
                            ymin=0.35, ymax=0.65
                        ]
                        \addplot [
                            hist={data min=0.35, data max=0.65, handler/.style={xbar interval}, bins=30}, % Use ybar interval
                            fill=gray!50,
                            x filter/.code=\pgfmathparse{rawy}, % Interpret the x values of the histogram as y values 
                            y filter/.code=\pgfmathparse{rawx} % And vice versa
                        ] table [col sep=comma, x=F1, y=wF1] {ressources/data/results3.csv};
                    \end{axis}
                \end{tikzpicture}
                \label{fig:prompt3}
            \end{subfigure}
            \hfill
            \begin{subfigure}[b]{0.45\textwidth}
                \centering
                \begin{tikzpicture}
                    % Main axis (scatter plot)
                    \begin{axis}[
                            enlargelimits=false,
                            title={Prompt 4},
                            title style={at={(0.2,1.2)}, anchor=north},
                            xlabel={$BERT(P_{4}^{z})$},
                            % ylabel={$BLEU(P')$},
                            xmin=0.35, xmax=0.65,
                            ymin=0.35, ymax=0.65,
                            xtick distance=0.1, ytick distance=0.1,
                            name=main axis,
                            width=\textwidth,
                            height=\textwidth
                        ]
                        \addplot[color=black, thick, dotted] {x};
                        \addplot+[
                            color=blue,
                            only marks,
                            mark=o,
                            mark size=1.0pt
                        ] table [col sep=comma, x=F1, y=wF1] {ressources/data/results4.csv};
                    \end{axis}

                    % Histogram for the x axis
                    \begin{axis}[
                            anchor=south west,
                            at={(main axis.north west)},
                            height=3cm,
                            width=\textwidth,
                            xtick=\empty,
                            ytick=\empty,
                            axis x line*=bottom,
                            axis y line*=left,
                            axis line style={draw=none},
                            xmin=0.35, xmax=0.65
                        ]
                        \addplot [
                            hist={data=x, bins=30}, % Increase number of bins for better resolution, % Use ybar interval
                            fill=gray!50
                        ] table [col sep=comma, x=F1, y=wF1] {ressources/data/results4.csv};
                    \end{axis}

                    % Histogram for the y axis
                    \begin{axis}[
                            anchor=north west,
                            at={(main axis.north east)},
                            width=3cm,
                            height=\textwidth,
                            xtick=\empty,
                            ytick=\empty,
                            axis x line*=top,
                            axis y line*=right,
                            axis line style={draw=none},
                            ymin=0.35, ymax=0.65
                        ]
                        \addplot [
                            hist={data min=0.35, data max=0.65, handler/.style={xbar interval}, bins=30}, % Use ybar interval
                            fill=gray!50,
                            x filter/.code=\pgfmathparse{rawy}, % Interpret the x values of the histogram as y values 
                            y filter/.code=\pgfmathparse{rawx} % And vice versa
                        ] table [col sep=comma, x=F1, y=wF1] {ressources/data/results4.csv};
                    \end{axis}
                \end{tikzpicture}
                \label{fig:prompt4}
            \end{subfigure}
        \end{minipage}
    }
    \caption{Blue-Metric of Zero-Shot Characterizations generated with Llama without ($P$) and with passage retrieval ($P'$) from the literature}
\end{figure}

\begin{figure}[H]
    \centering
    \makebox[\textwidth][c]{
        \begin{minipage}{1.1\textwidth}
            \centering
            \begin{subfigure}[b]{0.45\textwidth}
                \centering
                \begin{tikzpicture}
                    % Main axis (scatter plot)
                    \begin{axis}[
                            enlargelimits=false,
                            title={Prompt 1},
                            xlabel={$BERT(P_{1}^{z})$},
                            ylabel={$BERT(P^{r})$},
                            xmin=0, xmax=16,
                            ymin=0, ymax=16,
                            name=main axis,
                            width=\textwidth,
                            height=\textwidth
                        ]
                        \addplot[color=black, thick, dotted] {x};
                        \addplot+[
                            color=blue,
                            only marks,
                            mark=o,
                            mark size=1.0pt
                        ] table [col sep=comma, x=h, y=wh] {ressources/data/results.csv};
                    \end{axis}

                    % Histogram for the x axis
                    \begin{axis}[
                            anchor=south west,
                            at={(main axis.north west)},
                            height=3cm,
                            width=\textwidth,
                            xtick=\empty,
                            ytick=\empty,
                            axis x line*=bottom,
                            axis y line*=left,
                            axis line style={draw=none},
                            xmin=0, xmax=16,
                            ymin=0, ymax=400
                        ]
                        \addplot [
                            hist={data=x, bins=30}, % Increase number of bins for better resolution, % Use ybar interval
                            fill=gray!50
                        ] table [col sep=comma, x=h, y=wh] {ressources/data/results.csv};
                    \end{axis}

                    % Histogram for the y axis
                    \begin{axis}[
                            anchor=north west,
                            at={(main axis.north east)},
                            width=3cm,
                            height=\textwidth,
                            xtick=\empty,
                            ytick=\empty,
                            axis x line*=top,
                            axis y line*=right,
                            axis line style={draw=none},
                            ymin=0, ymax=16,
                            xmin=0, xmax=400
                        ]
                        \addplot [
                            hist={data min=0, data max=16, handler/.style={xbar interval}, bins=30}, % Use ybar interval
                            fill=gray!50,
                            x filter/.code=\pgfmathparse{rawy}, % Interpret the x values of the histogram as y values 
                            y filter/.code=\pgfmathparse{rawx} % And vice versa
                        ] table [col sep=comma, x=h, y=wh] {ressources/data/results.csv};
                    \end{axis}
                \end{tikzpicture}
                \label{fig:prompt1}
            \end{subfigure}
            \hfill
            \begin{subfigure}[b]{0.45\textwidth}
                \centering
                \begin{tikzpicture}
                    % Main axis (scatter plot)
                    \begin{axis}[
                            enlargelimits=false,
                            title={Prompt 1},
                            xlabel={$BERT(P_{2}^{z})$},
                            % ylabel={$BERT(P')$},
                            xmin=0, xmax=16,
                            ymin=0, ymax=16,
                            name=main axis,
                            width=\textwidth,
                            height=\textwidth
                        ]
                        \addplot[color=black, thick, dotted] {x};
                        \addplot+[
                            color=blue,
                            only marks,
                            mark=o,
                            mark size=1.0pt
                        ] table [col sep=comma, x=h, y=wh] {ressources/data/results2.csv};
                    \end{axis}

                    % Histogram for the x axis
                    \begin{axis}[
                            anchor=south west,
                            at={(main axis.north west)},
                            height=3cm,
                            width=\textwidth,
                            xtick=\empty,
                            ytick=\empty,
                            axis x line*=bottom,
                            axis y line*=left,
                            axis line style={draw=none},
                            xmin=0, xmax=16,
                            ymin=0, ymax=400
                        ]
                        \addplot [
                            hist={data=x, bins=30}, % Increase number of bins for better resolution, % Use ybar interval
                            fill=gray!50
                        ] table [col sep=comma, x=h, y=wh]{ressources/data/results2.csv};
                    \end{axis}

                    % Histogram for the y axis
                    \begin{axis}[
                            anchor=north west,
                            at={(main axis.north east)},
                            width=3cm,
                            height=\textwidth,
                            xtick=\empty,
                            ytick=\empty,
                            axis x line*=top,
                            axis y line*=right,
                            axis line style={draw=none},
                            ymin=0, ymax=16,
                            xmin=0, xmax=400
                        ]
                        \addplot [
                            hist={data min=0, data max=16, handler/.style={xbar interval}, bins=30}, % Use ybar interval % Increase number of bins for better resolution
                            fill=gray!50,
                            x filter/.code=\pgfmathparse{rawy}, % Interpret the x values of the histogram as y values 
                            y filter/.code=\pgfmathparse{rawx} % And vice versa
                        ] table [col sep=comma, x=h, y=wh] {ressources/data/results2.csv};
                    \end{axis}
                \end{tikzpicture}
                \label{fig:prompt2}
            \end{subfigure}
            \vspace{0.5cm}
            \begin{subfigure}[b]{0.45\textwidth}
                \centering
                \begin{tikzpicture}
                    % Main axis (scatter plot)
                    \begin{axis}[
                            enlargelimits=false,
                            title={Prompt 1},
                            xlabel={$BERT(P_{3}^{z})$},
                            ylabel={$BERT(P^{r})$},
                            xmin=0, xmax=16,
                            ymin=0, ymax=16,
                            name=main axis,
                            width=\textwidth,
                            height=\textwidth
                        ]
                        \addplot[color=black, thick, dotted] {x};
                        \addplot+[
                            color=blue,
                            only marks,
                            mark=o,
                            mark size=1.0pt
                        ] table [col sep=comma, x=h, y=wh] {ressources/data/results3.csv};
                    \end{axis}

                    % Histogram for the x axis
                    \begin{axis}[
                            anchor=south west,
                            at={(main axis.north west)},
                            height=3cm,
                            width=\textwidth,
                            xtick=\empty,
                            ytick=\empty,
                            axis x line*=bottom,
                            axis y line*=left,
                            axis line style={draw=none},
                            xmin=0, xmax=16,
                            ymin=0, ymax=400
                        ]
                        \addplot [
                            hist={data=x, bins=30}, % Increase number of bins for better resolution, % Use ybar interval
                            fill=gray!50
                        ] table [col sep=comma, x=h, y=wh] {ressources/data/results3.csv};
                    \end{axis}

                    % Histogram for the y axis
                    \begin{axis}[
                            anchor=north west,
                            at={(main axis.north east)},
                            width=3cm,
                            height=\textwidth,
                            xtick=\empty,
                            ytick=\empty,
                            axis x line*=top,
                            axis y line*=right,
                            axis line style={draw=none},
                            ymin=0, ymax=16,
                            xmin=0, xmax=400
                        ]
                        \addplot [
                            hist={data min=0, data max=16, handler/.style={xbar interval}, bins=30}, % Use ybar interval % Increase number of bins for better resolution
                            fill=gray!50,
                            x filter/.code=\pgfmathparse{rawy}, % Interpret the x values of the histogram as y values 
                            y filter/.code=\pgfmathparse{rawx} % And vice versa
                        ] table [col sep=comma, x=h, y=wh] {ressources/data/results3.csv};
                    \end{axis}
                \end{tikzpicture}
                \label{fig:prompt3}
            \end{subfigure}
            \hfill
            \begin{subfigure}[b]{0.45\textwidth}
                \centering
                \begin{tikzpicture}
                    % Main axis (scatter plot)
                    \begin{axis}[
                            enlargelimits=false,
                            title={Prompt 1},
                            xlabel={$BERT(P_{4}^{z})$},
                            % ylabel={$BERT(P')$},
                            xmin=0, xmax=16,
                            ymin=0, ymax=16,
                            name=main axis,
                            width=\textwidth,
                            height=\textwidth
                        ]
                        \addplot[color=black, thick, dotted] {x};
                        \addplot+[
                            color=blue,
                            only marks,
                            mark=o,
                            mark size=1.0pt
                        ] table [col sep=comma, x=h, y=wh] {ressources/data/results4.csv};
                    \end{axis}

                    % Histogram for the x axis
                    \begin{axis}[
                            anchor=south west,
                            at={(main axis.north west)},
                            height=3cm,
                            width=\textwidth,
                            xtick=\empty,
                            ytick=\empty,
                            axis x line*=bottom,
                            axis y line*=left,
                            axis line style={draw=none},
                            xmin=0, xmax=16,
                            ymin=0, ymax=400
                        ]
                        \addplot [
                            hist={data=x, bins=30}, % Increase number of bins for better resolution, % Use ybar interval
                            fill=gray!50
                        ] table [col sep=comma, x=h, y=wh] {ressources/data/results4.csv};
                    \end{axis}

                    % Histogram for the y axis
                    \begin{axis}[
                            anchor=north west,
                            at={(main axis.north east)},
                            width=3cm,
                            height=\textwidth,
                            xtick=\empty,
                            ytick=\empty,
                            axis x line*=top,
                            axis y line*=right,
                            axis line style={draw=none},
                            ymin=0, ymax=16,
                            xmin=0, xmax=400
                        ]
                        \addplot [
                            hist={data min=0, data max=16, handler/.style={xbar interval}, bins=30}, % Use ybar interval % Increase number of bins for better resolution
                            fill=gray!50,
                            x filter/.code=\pgfmathparse{rawy}, % Interpret the x values of the histogram as y values 
                            y filter/.code=\pgfmathparse{rawx} % And vice versa
                        ] table [col sep=comma, x=h, y=wh] {ressources/data/results4.csv};
                    \end{axis}
                \end{tikzpicture}
                \label{fig:prompt4}
            \end{subfigure}
        \end{minipage}
    }

    \caption{BERTScore of Zero-Shot Characterizations generated with Llama without ($P$) and with passage retrieval ($P'$) from the literatur}
\end{figure}




%Other




\begin{figure}[H]
    \centering
    \begin{subfigure}[b]{\textwidth}
        \centering
        \begin{tikzpicture}
            \pgfplotstableread[col sep=comma]{ressources/data/boxplot.csv}\csvdata
            \pgfplotstabletranspose\datatransposed{\csvdata}
            \begin{axis}[
                    boxplot/draw direction = y,
                    axis x line* = bottom,
                    axis y line = left,
                    enlarge y limits,
                    ymajorgrids,
                    cycle list={{blue}},
                    xtick = {1,2,3,4,5,6,7,8},
                    xticklabel style = {align=center, font=\small},
                    xticklabels = {$P_{1}$, $P_{1}'$, $P_{2}$, $P_{2}'$, $P_{3}$, $P_{3}'$, $P_{4}$, $P_{4}'$},
                    xtick style = {draw=none},
                    ylabel = {$BLEU$}
                ]
                \foreach \n in {1,...,8} {
                        \addplot+[boxplot, draw=black] table[y index=\n] {\datatransposed};
                    }
            \end{axis}
        \end{tikzpicture}
        \label{fig:boxplot}
    \end{subfigure}
    \hfill
    \vspace{1cm}
    \hfill
    \begin{subfigure}[b]{\textwidth}
        \centering
        \begin{tikzpicture}
            \begin{groupplot}[
                    group style={
                            group size=1 by 2,
                            vertical sep=0pt
                        },
                    boxplot/draw direction = y,
                    enlarge y limits,
                    ymajorgrids,
                    cycle list={{blue}},
                    xmin=0,
                    xmax=9,
                    height=10cm,
                    width=10cm
                ]
                \nextgroupplot[
                    hide x axis,
                    axis y line=left,
                    axis y discontinuity=parallel,
                    ymin=7,
                    ymax=18,
                    height=3.5cm,
                    y label style={at={(axis description cs:-0.1,.1)},rotate=0,anchor=south},
                    ylabel = ${BERT}$
                ]

                \addplot+[
                    boxplot prepared from table={
                            table=\csvdata,
                            row=0, % 0-based index
                            lower whisker=lw,
                            upper whisker=uw,
                            lower quartile=lq,
                            upper quartile=uq,
                            median=med
                        }, boxplot prepared,   draw=black
                ]
                coordinates {};

                \addplot+[
                    boxplot prepared from table={
                            table=\csvdata,
                            row=1, % 0-based index
                            lower whisker=lw,
                            upper whisker=uw,
                            lower quartile=lq,
                            upper quartile=uq,
                            median=med
                        }, boxplot prepared,   draw=black
                ]
                coordinates {};

                \addplot+[
                    boxplot prepared from table={
                            table=\csvdata,
                            row=2, % 0-based index
                            lower whisker=lw,
                            upper whisker=uw,
                            lower quartile=lq,
                            upper quartile=uq,
                            median=med
                        }, boxplot prepared,   draw=black
                ]
                coordinates {};
                \addplot+[
                    boxplot prepared from table={
                            table=\csvdata,
                            row=3, % 0-based index
                            lower whisker=lw,
                            upper whisker=uw,
                            lower quartile=lq,
                            upper quartile=uq,
                            median=med
                        }, boxplot prepared,   draw=black
                ]
                coordinates {};

                \addplot+[
                    boxplot prepared from table={
                            table=\csvdata,
                            row=4, % 0-based index
                            lower whisker=lw,
                            upper whisker=uw,
                            lower quartile=lq,
                            upper quartile=uq,
                            median=med
                        }, boxplot prepared,   draw=black
                ]
                coordinates {};

                \addplot+[
                    boxplot prepared from table={
                            table=\csvdata,
                            row=5, % 0-based index
                            lower whisker=lw,
                            upper whisker=uw,
                            lower quartile=lq,
                            upper quartile=uq,
                            median=med
                        }, boxplot prepared,   draw=black
                ]
                coordinates {};

                \addplot+[
                    boxplot prepared from table={
                            table=\csvdata,
                            row=6, % 0-based index
                            lower whisker=lw,
                            upper whisker=uw,
                            lower quartile=lq,
                            upper quartile=uq,
                            median=med
                        }, boxplot prepared,   draw=black
                ]
                coordinates {};

                \addplot+[
                    boxplot prepared from table={
                            table=\csvdata,
                            row=7, % 0-based index
                            lower whisker=lw,
                            upper whisker=uw,
                            lower quartile=lq,
                            upper quartile=uq,
                            median=med
                        }, boxplot prepared,   draw=black
                ]
                coordinates {};





                \nextgroupplot[
                    xtick = {1,2,3,4,5,6,7,8},
                    xticklabel style = {align=center, font=\small},
                    xticklabels = {$P_{1}$, $P_{1}'$, $P_{2}$, $P_{2}'$, $P_{3}$, $P_{3}'$, $P_{4}$, $P_{4}'$},
                    xtick style = {draw=none},
                    axis x line* = bottom,
                    axis y line = left,
                    ymin=0,
                    ymax=2,
                    height=3.5cm
                ]

                \addplot+[
                    boxplot prepared from table={
                            table=\csvdata,
                            row=0, % 0-based index
                            lower whisker=lw,
                            upper whisker=uw,
                            lower quartile=lq,
                            upper quartile=uq,
                            median=med
                        }, boxplot prepared,   draw=black
                ]
                coordinates {};

                \addplot+[
                    boxplot prepared from table={
                            table=\csvdata,
                            row=1, % 0-based index
                            lower whisker=lw,
                            upper whisker=uw,
                            lower quartile=lq,
                            upper quartile=uq,
                            median=med
                        }, boxplot prepared,   draw=black
                ]
                coordinates {};

                \addplot+[
                    boxplot prepared from table={
                            table=\csvdata,
                            row=2, % 0-based index
                            lower whisker=lw,
                            upper whisker=uw,
                            lower quartile=lq,
                            upper quartile=uq,
                            median=med
                        }, boxplot prepared,   draw=black
                ]
                coordinates {};
                \addplot+[
                    boxplot prepared from table={
                            table=\csvdata,
                            row=3, % 0-based index
                            lower whisker=lw,
                            upper whisker=uw,
                            lower quartile=lq,
                            upper quartile=uq,
                            median=med
                        }, boxplot prepared,   draw=black
                ]
                coordinates {};

                \addplot+[
                    boxplot prepared from table={
                            table=\csvdata,
                            row=4, % 0-based index
                            lower whisker=lw,
                            upper whisker=uw,
                            lower quartile=lq,
                            upper quartile=uq,
                            median=med
                        }, boxplot prepared,   draw=black
                ]
                coordinates {};

                \addplot+[
                    boxplot prepared from table={
                            table=\csvdata,
                            row=5, % 0-based index
                            lower whisker=lw,
                            upper whisker=uw,
                            lower quartile=lq,
                            upper quartile=uq,
                            median=med
                        }, boxplot prepared,   draw=black
                ]
                coordinates {};

                \addplot+[
                    boxplot prepared from table={
                            table=\csvdata,
                            row=6, % 0-based index
                            lower whisker=lw,
                            upper whisker=uw,
                            lower quartile=lq,
                            upper quartile=uq,
                            median=med
                        }, boxplot prepared,   draw=black
                ]
                coordinates {};

                \addplot+[
                    boxplot prepared from table={
                            table=\csvdata,
                            row=7, % 0-based index
                            lower whisker=lw,
                            upper whisker=uw,
                            lower quartile=lq,
                            upper quartile=uq,
                            median=med
                        }, boxplot prepared,   draw=black
                ]
                coordinates {};


            \end{groupplot}
        \end{tikzpicture}

        \label{fig:boxplot}
    \end{subfigure}
    \vspace{1cm}
    \caption{Boxplots of BLEU- and BERTScores for every prompt ($P_{1}, P_{1}', P_{2},...,P_{4}'$)}
\end{figure}

\begin{figure}[H]
    \begin{center}
        \begin{tabular}{|c|c||c|c|c|c|}
            \hline
            Prompt  & Heuristic & T-Test            & Spearman Correlation \\ [0.5ex]
            \hline\hline
            $P_{1}$ & BLEU      & -15.0 (6.65e-40)  & 0.62 (3.46e-40)      \\
            $P_{1}$ & BERT      & -3.04 (2.56e-03)  & 0.91 (4.34e-142)     \\
            $P_{2}$ & BLEU      & -4.2 (3.08e-05)   & 0.74 (1.23e-95)      \\
            $P_{2}$ & BERT      & -2.58 (1.00e-02)  & 0.92 (5.12e-224)     \\
            $P_{3}$ & BLEU      & -15.52 (4.30e-45) & 0.7 (1.76e-79)       \\
            $P_{3}$ & BERT      & -2.06 (3.94e-02)  & 0.92 (2.51e-218)     \\
            $P_{4}$ & BLEU      & -20.65 (4.30e-70) & 0.61 (6.66e-57)      \\
            $P_{4}$ & BERT      & -4.0 (7.13e-05)   & 0.92 (2.14e-220)     \\
            \hline\hline
        \end{tabular}
    \end{center}
    \label{fig:prompts2}
    \caption{T-Test and spearman correlation with according p-values after prompting llama3 and gemma2 with $P_{1}$ with and without a selection of embedded text chunks for passage retrieval}
\end{figure}



\subsection{Analysis}
Having a more detailed look at the results, we can see that the BLEU scores of each prompt mostly improve after passage retrieval. Although the maximum values of $P_{1}^{z}$ and $P_{2}^{z}$ have decreased slightly in $P_{1}^{r}$ and $P_{2}^{r}$, the minimum values, Q1, and Q3 have significantly increased, as observed in the box plots. For both $P_{3}^{z}$ and $P_{4}^{z}$, every box plot quartile has improved.\\

For BERTScore, the improvement isn't quite as visible. In fact, the upper quartiles have a lower maximum after passage retrieval, but Q1-Q3 has improved slightly for every prompt. Consequently, the results are more compact. Some outliers close to the maximum in Q4 might score so high prior to passage retrieval due to Llama being trained on similar information to the fandom articles. Especially when generating summaries for main characters, Llama might already have a great knowledge base for that character, and relying solely on the additional passed sentences might therefore be hindering in generating a good characterization.\\

Obviously the method of passage retrieval used for this experiment isn't ideal, as regularly eliminating sentences could omit important context, also at this stage, the process of fetching fandom articles wasn't complete, resulting in a dataset with some duplicates and missing characterizations. Despite these limitations, the data is still sufficient enough to show two important aspects. First, the results with passage retrieval are at least as good as, or already slightly better than, those without. The similarity of the vocabulary has increased quite significantly whereas semantics seem only to have improved slightly.

Second, the results and the different wordings in the prompts definately have an influence on the results average and variance \ref*{fig:enter-label}. Nevertheless choosing the right prompt for this task isn't as simple as choosing the results with the highest score average, a low variance is even more crucial since testifies a higher precision and is therefore a more accurate prompt for archieving the desired output. Based on this deduction and the observations of BERT- and BLEUScore i will continue the next experiment with prompt $P_{2}$.\\since it had the highest results in both metrics and will improve on the method for passage retrieval.


\newpage
\section{Selected Embedded chunks}

We will now continue with $P_{1}$ from the base experiment. Instead of selecting $n$ sentences that just containt the name, we first chunk each books into roughly 1000 character big chunks and then use BERT to create embeddings. Then we are trying to retrieve the chunks that describe the character best.



\begin{figure}[H]
    \centering
    \makebox[\textwidth][c]{
        \begin{minipage}{1.1\textwidth}
            \centering
            \begin{subfigure}[b]{0.45\textwidth}
                \centering
                \begin{tikzpicture}
                    % Main axis (scatter plot)
                    \begin{axis}[
                            enlargelimits=false,
                            title={(Llama3)},
                            title style={at={(0.1,1.2)}, anchor=north},
                            xlabel={$BLEU(P_{1}^{z})$},
                            ylabel={$BLEU(P^{r})$},
                            xmin=0.35, xmax=0.65,
                            ymin=0.35, ymax=0.65,
                            xtick distance=0.1, ytick distance=0.1,
                            name=main axis,
                            width=\textwidth,
                            height=\textwidth
                        ]
                        \addplot[color=black, thick, dotted] {x};
                        \addplot+[
                            color=blue,
                            only marks,
                            mark=o,
                            mark size=1.0pt
                        ] table [col sep=comma, x=F1, y=wF1] {ressources/data/normal_eval.csv};
                    \end{axis}

                    % Histogram for the x axis
                    \begin{axis}[
                            anchor=south west,
                            at={(main axis.north west)},
                            height=3cm,
                            width=\textwidth,
                            xtick=\empty,
                            ytick=\empty,
                            axis x line*=bottom,
                            axis y line*=left,
                            axis line style={draw=none},
                            xmin=0.35, xmax=0.65
                        ]
                        \addplot [
                            hist={data=x, bins=30}, % Increase number of bins for better resolution, % Use ybar interval
                            fill=gray!50
                        ] table [col sep=comma, x=F1, y=wF1] {ressources/data/normal_eval.csv};
                    \end{axis}

                    % Histogram for the y axis
                    \begin{axis}[
                            anchor=north west,
                            at={(main axis.north east)},
                            width=3cm,
                            height=\textwidth,
                            xtick=\empty,
                            ytick=\empty,
                            axis x line*=top,
                            axis y line*=right,
                            axis line style={draw=none},
                            ymin=0.35, ymax=0.65
                        ]
                        \addplot [
                            hist={data min=0.35, data max=0.65, handler/.style={xbar interval}, bins=30}, % Use ybar interval
                            fill=gray!50,
                            x filter/.code=\pgfmathparse{rawy}, % Interpret the x values of the histogram as y values
                            y filter/.code=\pgfmathparse{rawx} % And vice versa
                        ] table [col sep=comma, x=F1, y=wF1] {ressources/data/normal_eval.csv};
                    \end{axis}
                \end{tikzpicture}
                \label{fig:prompt1}
            \end{subfigure}
            \hfill
            \begin{subfigure}[b]{0.45\textwidth}
                \centering
                \begin{tikzpicture}
                    % Main axis (scatter plot)
                    \begin{axis}[
                            enlargelimits=false,
                            title={(Gemma2)},
                            title style={at={(0.1,1.2)}, anchor=north},
                            xlabel={$BLEU(P_{1}^{z})$},
                            % ylabel={$BLEU(P')$},
                            xmin=0.35, xmax=0.65,
                            ymin=0.35, ymax=0.65,
                            xtick distance=0.1, ytick distance=0.1,
                            name=main axis,
                            width=\textwidth,
                            height=\textwidth
                        ]
                        \addplot[color=black, thick, dotted] {x};
                        \addplot+[
                            color=blue,
                            only marks,
                            mark=o,
                            mark size=1.0pt
                        ] table [col sep=comma, x=F1, y=wF1] {ressources/data/normal_eval2.csv};
                    \end{axis}

                    % Histogram for the x axis
                    \begin{axis}[
                            anchor=south west,
                            at={(main axis.north west)},
                            height=3cm,
                            width=\textwidth,
                            xtick=\empty,
                            ytick=\empty,
                            axis x line*=bottom,
                            axis y line*=left,
                            axis line style={draw=none},
                            xmin=0.35, xmax=0.65
                        ]
                        \addplot [
                            hist={data=x, bins=30},
                            fill=gray!50
                        ] table [col sep=comma, x=F1, y=wF1] {ressources/data/normal_eval2.csv};
                    \end{axis}

                    % Histogram for the y axis
                    \begin{axis}[
                            anchor=north west,
                            at={(main axis.north east)},
                            width=3cm,
                            height=\textwidth,
                            xtick=\empty,
                            ytick=\empty,
                            axis x line*=top,
                            axis y line*=right,
                            axis line style={draw=none},
                            ymin=0.35, ymax=0.65
                        ]
                        \addplot [
                            hist={data min=0.35, data max=0.65, handler/.style={xbar interval}, bins=30}, % Use ybar interval
                            fill=gray!50,
                            x filter/.code=\pgfmathparse{rawy}, % Interpret the x values of the histogram as y values
                            y filter/.code=\pgfmathparse{rawx} % And vice versa
                        ] table [col sep=comma, x=F1, y=wF1] {ressources/data/normal_eval2.csv};
                    \end{axis}
                \end{tikzpicture}
                \label{fig:prompt2}
            \end{subfigure}
        \end{minipage}
    }
    \caption{Blue-Metric of Zero-Shot Characterizations generated with Llama3 and Gemma2 without ($P$) and with passage retrieval ($P'$) from the literature}
\end{figure}

\begin{figure}[H]
    \centering
    \makebox[\textwidth][c]{
        \begin{minipage}{1.1\textwidth}
            \centering
            \begin{subfigure}[b]{0.45\textwidth}
                \centering
                \begin{tikzpicture}
                    % Main axis (scatter plot)
                    \begin{axis}[
                            enlargelimits=false,
                            title={(Llama3)},
                            xlabel={$BERT(P_{1}^{z})$},
                            ylabel={$BERT(P^{r})$},
                            xmin=0, xmax=16,
                            ymin=0, ymax=16,
                            name=main axis,
                            width=\textwidth,
                            height=\textwidth
                        ]
                        \addplot[color=black, thick, dotted] {x};
                        \addplot+[
                            color=blue,
                            only marks,
                            mark=o,
                            mark size=1.0pt
                        ] table [col sep=comma, x=h, y=wh] {ressources/data/normal_eval.csv};
                    \end{axis}

                    % Histogram for the x axis
                    \begin{axis}[
                            anchor=south west,
                            at={(main axis.north west)},
                            height=3cm,
                            width=\textwidth,
                            xtick=\empty,
                            ytick=\empty,
                            axis x line*=bottom,
                            axis y line*=left,
                            axis line style={draw=none},
                            xmin=0, xmax=16,
                            ymin=0, ymax=400
                        ]
                        \addplot [
                            hist={data=x, bins=30},
                            fill=gray!50
                        ] table [col sep=comma, x=h, y=wh] {ressources/data/normal_eval.csv};
                    \end{axis}

                    % Histogram for the y axis
                    \begin{axis}[
                            anchor=north west,
                            at={(main axis.north east)},
                            width=3cm,
                            height=\textwidth,
                            xtick=\empty,
                            ytick=\empty,
                            axis x line*=top,
                            axis y line*=right,
                            axis line style={draw=none},
                            ymin=0, ymax=16,
                            xmin=0, xmax=400
                        ]
                        \addplot [
                            hist={data min=0, data max=16, handler/.style={xbar interval}, bins=30}, % Use ybar interval
                            fill=gray!50,
                            x filter/.code=\pgfmathparse{rawy}, % Interpret the x values of the histogram as y values 
                            y filter/.code=\pgfmathparse{rawx} % And vice versa
                        ] table [col sep=comma, x=h, y=wh] {ressources/data/normal_eval.csv};
                    \end{axis}
                \end{tikzpicture}
                \label{fig:prompt1}
            \end{subfigure}
            \hfill
            \begin{subfigure}[b]{0.45\textwidth}
                \centering
                \begin{tikzpicture}
                    % Main axis (scatter plot)
                    \begin{axis}[
                            enlargelimits=false,
                            title={(Gemma2)},
                            xlabel={$BERT(P_{1}^{z})$},
                            % ylabel={$BERT(P')$},
                            xmin=0, xmax=16,
                            ymin=0, ymax=16,
                            name=main axis,
                            width=\textwidth,
                            height=\textwidth
                        ]
                        \addplot[color=black, thick, dotted] {x};
                        \addplot+[
                            color=blue,
                            only marks,
                            mark=o,
                            mark size=1.0pt
                        ] table [col sep=comma, x=h, y=wh] {ressources/data/normal_eval2.csv};
                    \end{axis}

                    % Histogram for the x axis
                    \begin{axis}[
                            anchor=south west,
                            at={(main axis.north west)},
                            height=3cm,
                            width=\textwidth,
                            xtick=\empty,
                            ytick=\empty,
                            axis x line*=bottom,
                            axis y line*=left,
                            axis line style={draw=none},
                            xmin=0, xmax=16,
                            ymin=0, ymax=400
                        ]
                        \addplot [
                            hist={data=x, bins=30}, % Increase number of bins for better resolution
                            fill=gray!50
                        ] table [col sep=comma, x=h, y=wh]{ressources/data/normal_eval2.csv};
                    \end{axis}

                    % Histogram for the y axis
                    \begin{axis}[
                            anchor=north west,
                            at={(main axis.north east)},
                            width=3cm,
                            height=\textwidth,
                            xtick=\empty,
                            ytick=\empty,
                            axis x line*=top,
                            axis y line*=right,
                            axis line style={draw=none},
                            ymin=0, ymax=16,
                            xmin=0, xmax=400
                        ]
                        \addplot [
                            hist={data min=0, data max=16, handler/.style={xbar interval}, bins=30}, % Use ybar interval
                            fill=gray!50,
                            x filter/.code=\pgfmathparse{rawy}, % Interpret the x values of the histogram as y values 
                            y filter/.code=\pgfmathparse{rawx} % And vice versa
                        ] table [col sep=comma, x=h, y=wh] {ressources/data/normal_eval2.csv};
                    \end{axis}
                \end{tikzpicture}
                \label{fig:prompt2}
            \end{subfigure}
        \end{minipage}
    }

    \caption{BERTScore of Zero-Shot Characterizations generated with Llama3 and Gemma2 without ($P$) and with passage retrieval ($P'$) from the literature}
\end{figure}




\begin{figure}[H]
    \begin{center}
        \begin{tabular}{|c|c||c|c|c|c|}
            \hline
            Model  & Heuristic & T-Test            & Spearman Correlation \\ [0.5ex]
            \hline\hline
            llama3 & BLEU      & -10.31 (6.99e-23) & 0.55 (2.40e-43)      \\
            \hline
            llama3 & BERT      & -7.42 (4.52e-13)  & 0.77 (1.03e-106)     \\
            \hline
            Gemma2 & BLEU      & -14.29 (1.64e-39) & 0.53 (7.09e-40)      \\
            \hline
            Gemma2 & BERT      & -3.69 (2.43e-4)   & 0.56 (6.30e-46)      \\
            \hline
            \hline
        \end{tabular}
    \end{center}
    \label{fig:prompts2}
    \caption{T-Test and spearman correlation with according p-values after prompting llama3 and gemma2 with $P_{1}$ with and without a selection of embedded text chunks for passage retrieval}
\end{figure}


