\chapter{Related Work}
Now first of all there already has been a decent amount of approaches for automatic text-summarization.
One of the oldest and most cited papers belongs to "Automatic Text Summarization Using a Machine Learning Approach" \cite{10.1007/3-540-36127-8_20}. It describes a summarization procedure based on trainable Machine Learning algorithms. Creating a Characterization is quite similar to making a Summarization of character related content but could also include deductions made from the behavior of that character. 
A recent Paper from 2021 \cite{brahman-etal-2021-characters-tell} presents a dataset called LiSCU (Literary Summaries with Character Understanding) that aims to facilitate research in character-centric narrative understanding. They used techniques for Character Identification, where the goal is to identify a character's name from an anonymized description, and Character Description Generation, which involves generating a description for a given character based on a literature summary. In contrast to their approach for Character Description Generation which required modeling long-range dependencies, I am using
Retrieval-augmented generation (RAG), which is a technique to improve the quality of LLM-generated responses by grounding the model on external sources. LLMs are inconsistent in terms of producing same quality responses for each and every topic, since they knowledge is based on finite amount of information, that isn't equally distributed for every potential topic. But Retrieval-augmented generation doesn't only reduce the need for internal sources (continuous training, lowering computational and financial costs) but also ensures that the model has access to the most current, reliable facts.
In this thesis I am primarily focusing on getting those important properties and behavior (key features) from the characters described in the literature to achieve better characterizations with grounded models that utilize this external information.